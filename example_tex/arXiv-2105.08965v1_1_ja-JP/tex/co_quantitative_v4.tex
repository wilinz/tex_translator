% 必要なパッケージをドキュメントの前文に追加してください:
% \usepackage{booktabs}

\newcommand{\R}{\textcolor{red}}
\newcommand{\B}{\textcolor{blue}}

\begin{table}[]
\centering
{\small
\begin{tabular}{@{}llll@{}}
\toprule
\multicolumn{1}{c}{\multirow{2}{*}{方法}} & \multicolumn{1}{c}{~\emph{ボート} w/}   & \multicolumn{1}{c}{~\emph{列車} w/}  & \multicolumn{1}{c}{~\emph{列車} w/}  \\
& \multicolumn{1}{c}{~\emph{水}} & \multicolumn{1}{c}{~\emph{鉄道}}          & \multicolumn{1}{c}{~\emph{プラットフォーム}}  \\ \midrule
\multicolumn{1}{l}{CAM~\cite{zhou2016learning}\textsubscript{CVPR'16}}              & \B{0.74} (33.1)   & \B{0.11} (52.9)   & \multicolumn{1}{l}{\B{0.09} (49.6)}   \\
\multicolumn{1}{l}{SEAM~\cite{wang2020self}\textsubscript{CVPR'20}}                 & \B{1.13} (30.7)   & \B{0.24} (48.6)   & \multicolumn{1}{l}{\B{0.20} (45.5)}   \\
\multicolumn{1}{l}{ICD~\cite{fan2020learning}\textsubscript{CVPR'20}}               & \B{0.47} (41.4)   & \B{0.11} (56.7)   & \multicolumn{1}{l}{\B{0.09} (49.2)}   \\
\multicolumn{1}{l}{SGAN~\cite{yao2020saliency}\textsubscript{ACCESS'20}}            & \B{0.10} (42.3)   & \B{0.02} (48.8)   & \multicolumn{1}{l}{\B{0.01} (36.3)}   \\
\multicolumn{1}{l}{我々のEPS}                                                         & \B{0.10} (55.0)   & \B{0.02} (78.1)   & \multicolumn{1}{l}{\B{0.01} (73.0)}   \\ \bottomrule
\end{tabular}
}
\vspace{2mm}
\caption{共起問題を扱う代表的な既存の方法との比較。各エントリは\B{青}で示された{$m_{k,c}$}(低いほど良い)と括弧内のIoU(高いほど良い)です。} \vspace{-2mm}
\label{tab:co_quantitative_v4}

\end{table}
